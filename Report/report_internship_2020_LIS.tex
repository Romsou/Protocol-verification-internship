\documentclass[12pt, a4paper]{report}
\usepackage[utf8]{inputenc}
%\usepackage[scaled]{helvet}
%\renewcommand*\familydefault{\sfdefault}
\usepackage[french]{babel}
\usepackage{geometry}
\usepackage{fancyhdr}
\usepackage[svgnames]{xcolor}
\usepackage{enumitem}
\usepackage{algorithm}
\usepackage{algpseudocode}
\usepackage{amsmath}
\usepackage{amssymb}
\usepackage{sectsty}


% Permet de changer la numérotation de la documentclass pour les sections
\renewcommand{\thesection}{\arabic{section}}

% Définition des listes
\frenchbsetup{StandardLists=true}
\setitemize{font=\color{SeaGreen}}

% Définition des marges (paquets geometry)
\geometry{margin=2.5cm, vmargin=2.5cm}

% Définition de l'indentation
\setlength{\parindent}{0cm}

% Couleur des sections et sous-sections
\sectionfont{\color{blue}}
\subsectionfont{\color{blue}}

\title{\color{blue}Internship Report: Protocol verification}
\author{Romain Soumard}
\date{\today}

\begin{document}
\maketitle

\section{Introduction}

Coming...

\section{Organisation presentation}

Coming...

\section{Study of the properties of protocols}

\subsection{Preparations}

In computer science, verification refers to a discipline that uses formal methods to study the properties of systems and check whether they fullfill certain specifications.

During the course of this semester, I was assigned with the study of the protocols used in the belenios system. Belenios is an electronic vote sytem which was developed by the researchers from INRIA (\color{red} préciser l'acronyme\color{black}). Before being able to take care of the task at hand, I  had to learn the tools used in this field of research.

Throughout my studies and researches, I have learnt the existence of several mathematical and logical abstract tools, techniques and properties:
\begin{itemize}
\item The use of cryptographic primitives, modeled by an equational theory
\item Process algebras (or process calculi), use to model concurrent systems, especially, applied Pi-calculus.
\item The use of prooftrees.
\end{itemize}

These tools allow us to prove a wide array of protocol properties, such as deducibility and authentifcation. You can, for example, use cryptographic primitives and pi-calculus to prove the property of authentification, or use prooftrees to prove deducibility, which is intuitively a property that refers to the ability to deduce a term from a set of terms. 

\section{Conclusion}

Coming...

\section{bibliography}


Coming...
\end{document}